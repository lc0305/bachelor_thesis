\begin{lstlisting}[caption={Van Eerd's "An Interesting Lock-free Queue"},label=append_mpmc]
#ifndef __MPMC_RING_H__
#define __MPMC_RING_H__

/**
    * Inspired by 'An Interesting Lock-free Queue'
    * Talk by Tony Van Eerd at the CppCon 2017:
    * https://www.youtube.com/watch?v=HP2InVqgBFM
    * Author: Lucas Craemer
    * */

#include <algorithm>
#include <array>
#include <atomic>
#include <cstddef>
#include <cstdint>
#include <optional>

// This is the entry structure for the buffer.
// It holds a pointer to the actual value and a generation for that pointer.
template <typename T> struct PtrGen {
    std::optional<T *> ptr;
    std::uint64_t gen;

    inline bool is_empty(std::uint64_t cmp_gen) noexcept {
        // The entry is empty if the ptr points to NULL and it is in the current
        // gen.
        return !ptr.has_value() && gen == cmp_gen;
    }

    inline bool is_valid(std::uint64_t cmp_gen) noexcept {
        // The entry is valid if the ptr does not point to NULL and it is in the
        // current gen.
        return ptr.has_value() && gen == cmp_gen;
    }
};

// This is the data structure for the head and tail of the queue.
// It holds an index for head/tail into the buffer (this index is likely close
// the real index) and the the generation of that index to mitigate the ABA
// problem.
template <std::size_t N> struct IdxGen {
    std::size_t idx;
    std::uint64_t gen;

    inline void incr() noexcept {
        // Increment the index, if the index reaches the buffer size start at 0
        // again and increment the generation.
        if (++idx == N) {
            idx = 0;
            ++gen;
        }
    }

    inline bool operator<(const IdxGen<N> &rhs) noexcept {
        return gen < rhs.gen || (gen == rhs.gen && idx < rhs.idx);
    }
};

// Multiple-producer multiple-consumer circular FIFO queue
// Stores pointer to a type T.
// N is the cacapcity of the queue.
template <typename T, std::size_t N> class MPMCRing {
private:
    std::atomic<IdxGen<N>> m_headish;
    // The buff in between head and tail, so that depending on the buff size they
    // will likely end up on a different cache line. (no contention on the cache
    // line)
    std::array<std::atomic<PtrGen<T>>, N> m_buff;
    std::atomic<IdxGen<N>> m_tailish;

public:
    constexpr MPMCRing() noexcept
        : m_headish(IdxGen<N>{0, 0}), m_tailish(IdxGen<N>{0, 0}) {
        // Start by zeroing the buffer.
        std::fill(std::begin(m_buff), std::end(m_buff), PtrGen<T>{std::nullopt, 0});
    }

    // return ring size
    constexpr std::size_t size() noexcept { return N; }

    // Enqueue a pointer to a type T.
    // Returns true on success.
    // Returns false if the queue is full and enqueueing was not successful.
    bool enqueue(T *val) noexcept {
        std::uint64_t prev_gen = 0;
        IdxGen<N> old_tailish;
        // Load the 'current tail'. This tail is just a hint
        // and does not have to be the 'real' tail.
        // No memory barrier required due to acquire-release CAS (I think).
        IdxGen<N> new_tailish = old_tailish =
            m_tailish.load(std::memory_order_relaxed);
        PtrGen<T> current_entry;
        do {
            // Iterate over the buffer while the entry is not empty.
            // No memory fencing required.
            while (!(current_entry =
                        m_buff[new_tailish.idx].load(std::memory_order_relaxed))
                        .is_empty(new_tailish.gen)) {
                // The entry is not empty because the pointer is either not NULL
                // or not in the same generation as the current tail.
                if (current_entry.gen < prev_gen) {
                    // The buffer is full, because the entry is not empty and the
                    // generation of the current entry has decreased.
                    // Update the tail if its smaller and return.
                    if (old_tailish < new_tailish) {
                        // Update the tail if its smaller and return.
                        // Strong CAS, because not in a loop.
                        // No writes, no memory barrier required.
                        m_tailish.compare_exchange_strong(old_tailish, new_tailish,
                                                            std::memory_order_relaxed,
                                                            std::memory_order_relaxed);
                    }
                    return false;
                }
                // Increment the tail for further iteration.
                new_tailish.incr();
                // Replace the previous generation with the generation of the
                // current entry if the pointer points to a valid address.
                if (current_entry.ptr.has_value())
                    prev_gen = current_entry.gen;
            }
            // Finally: Try adding the entry to the buffer. If another thread
            // changed the entry retry. Weak CAS, because loop.
            // Do not reorder any read or writes before/after the successful CAS.
        } while (!m_buff[new_tailish.idx].compare_exchange_weak(
            current_entry, PtrGen<T>{val, new_tailish.gen},
            std::memory_order_release, std::memory_order_acquire));
        // Increment the tail, a new value was added.
        new_tailish.incr();
        // Update the tail.
        // Strong CAS, because not in a loop.
        // No memory barrier required due to acquire-release CAS (I think).
        m_tailish.compare_exchange_strong(old_tailish, new_tailish,
                                            std::memory_order_relaxed,
                                            std::memory_order_relaxed);
        // Success.
        return true;
    }

    // Returns an optional pointer to a value T. If the queue is empty return
    // NULL.
    std::optional<T *> dequeue() noexcept {
        IdxGen<N> old_headish;
        // Load the 'current head'. This head is just a hint and
        // does not have to be the 'real' head.
        // No memory barrier required due to acquire-release  CAS (I think).
        IdxGen<N> new_headish = old_headish =
            m_headish.load(std::memory_order_relaxed);
        PtrGen<T> current_entry;

        do {
            // Iterate over the buffer while the entry is not valid.
            // No memory fencing required.
            while (!(current_entry =
                        m_buff[new_headish.idx].load(std::memory_order_relaxed))
                        .is_valid(new_headish.gen)) {
                // The entry is not valid because the pointer is either NULL or not in
                // the same generation as the current head.
                if (current_entry.gen == new_headish.gen) {
                    // If the entry is in the same generation as the current head
                    // the queue is empty.
                    if (old_headish < new_headish) {
                        // Update the head if its smaller and return.
                        // Strong CAS, because not in a loop.
                        // No writes, no memory barrier required.
                        m_headish.compare_exchange_strong(old_headish, new_headish,
                                                            std::memory_order_relaxed,
                                                            std::memory_order_relaxed);
                    }
                    return std::nullopt;
                }
                // Increment the head for further iteration.
                new_headish.incr();
            }
            // Finally: Try removing the entry from the buffer. If another thread
            // changed the entry retry. Weak CAS, because loop.
            // Do not reorder any read or writes before/after the successful CAS.
        } while (!m_buff[new_headish.idx].compare_exchange_weak(
            current_entry, PtrGen<T>{std::nullopt, (new_headish.gen + 1)},
            std::memory_order_release, std::memory_order_acquire));
        // Increment the head, a new value was removed from the buffer.
        new_headish.incr();
        // Update the head.
        // Strong CAS, because not in a loop.
        // No memory barrier required due to acquire-release CAS (I think).
        m_headish.compare_exchange_strong(old_headish, new_headish,
                                            std::memory_order_relaxed,
                                            std::memory_order_relaxed);
        // Success. Return the pointer.
        return current_entry.ptr;
    }
};

#endif
\end{lstlisting}