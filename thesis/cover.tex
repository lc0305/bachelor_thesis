% -*-latex-*-
% 
% For questions, comments, concerns or complaints:
% thesis@mit.edu
% 
%
% $Log: cover.tex,v $
% Revision 1.8  2008/05/13 15:02:15  jdreed
% Degree month is June, not May.  Added note about prevdegrees.
% Arthur Smith's title updated
%
% Revision 1.7  2001/02/08 18:53:16  boojum
% changed some \newpages to \cleardoublepages
%
% Revision 1.6  1999/10/21 14:49:31  boojum
% changed comment referring to documentstyle
%
% Revision 1.5  1999/10/21 14:39:04  boojum
% *** empty log message ***
%
% Revision 1.4  1997/04/18  17:54:10  othomas
% added page numbers on abstract and cover, and made 1 abstract
% page the default rather than 2.  (anne hunter tells me this
% is the new institute standard.)
%
% Revision 1.4  1997/04/18  17:54:10  othomas
% added page numbers on abstract and cover, and made 1 abstract
% page the default rather than 2.  (anne hunter tells me this
% is the new institute standard.)
%
% Revision 1.3  93/05/17  17:06:29  starflt
% Added acknowledgements section (suggested by tompalka)
% 
% Revision 1.2  92/04/22  13:13:13  epeisach
% Fixes for 1991 course 6 requirements
% Phrase "and to grant others the right to do so" has been added to 
% permission clause
% Second copy of abstract is not counted as separate pages so numbering works
% out
% 
% Revision 1.1  92/04/22  13:08:20  epeisach

% NOTE:
% These templates make an effort to conform to the MIT Thesis specifications,
% however the specifications can change.  We recommend that you verify the
% layout of your title page with your thesis advisor and/or the MIT 
% Libraries before printing your final copy.
\title{How can Highly Concurrent Network-Bound Applications benefit from modern multi-core CPUs?}

\author{Lucas Crämer}
% If you wish to list your previous degrees on the cover page, use the 
% previous degrees command:
%       \prevdegrees{A.A., Harvard University (1985)}
% You can use the \\ command to list multiple previous degrees
%       \prevdegrees{B.S., University of California (1978) \\
%                    S.M., Massachusetts Institute of Technology (1981)}
\department{Fakultät Print und Medien}

% If the thesis is for two degrees simultaneously, list them both
% separated by \and like this:
% \degree{Doctor of Philosophy \and Master of Science}
\degree{Bachelor of Science in Mobile Media}

% As of the 2007-08 academic year, valid degree months are September, 
% February, or June.  The default is June.
\degreemonth{February}
\degreeyear{2021}
\thesisdate{February 28, 2021}

%% By default, the thesis will be copyrighted to MIT.  If you need to copyright
%% the thesis to yourself, just specify the `vi' documentclass option.  If for
%% some reason you want to exactly specify the copyright notice text, you can
%% use the \copyrightnoticetext command.  
\copyrightnoticetext{\copyright Hochschule der Medien, 2021}

% If there is more than one supervisor, use the \supervisor command
% once for each.
\supervisor{Prof. Walter Kriha}{Erstprüfer}
\supervisor{Dr. Thomas Fankhauser}{Zweitprüfer}

% This is the department committee chairman, not the thesis committee
% chairman.  You should replace this with your Department's Committee
% Chairman.
\chairman{Sibel Aktikkalmaz}{Prüfungsverwaltung Hochschule der Medien}

% Make the titlepage based on the above information.  If you need
% something special and can't use the standard form, you can specify
% the exact text of the titlepage yourself.  Put it in a titlepage
% environment and leave blank lines where you want vertical space.
% The spaces will be adjusted to fill the entire page.  The dotted
% lines for the signatures are made with the \signature command.
\maketitle

% The abstractpage environment sets up everything on the page except
% the text itself.  The title and other header material are put at the
% top of the page, and the supervisors are listed at the bottom.  A
% new page is begun both before and after.  Of course, an abstract may
% be more than one page itself.  If you need more control over the
% format of the page, you can use the abstract environment, which puts
% the word "Abstract" at the beginning and single spaces its text.

%% You can either \input (*not* \include) your abstract file, or you can put
%% the text of the abstract directly between the \begin{abstractpage} and
%% \end{abstractpage} commands.

% First copy: start a new page, and save the page number.
\cleardoublepage
% Uncomment the next line if you do NOT want a page number on your
% abstract and acknowledgments pages.
%\pagestyle{empty}
\cleardoublepage
%\pagestyle{empty}
\section*{Erklärung}

Hiermit versichere ich, Lucas Crämer, ehrenwörtlich, dass ich die vorliegende Bachelorarbeit mit dem Titel: "How can Highly Concurrent Network-Bound Applications benefit from modern multi-core CPUs?" (Forschungsfrage: "Wie können netzwerkgebundene Anwendungen von modernen Mehrkernprozessoren profitieren?") selbstständig und ohne fremde Hilfe verfasst und keine anderen als die angegebenen Hilfsmittel benutzt habe. Die Stellen der Arbeit, die dem Wortlaut oder dem Sinn nach anderen Werken entnommen wurden, sind in jedem Fall unter Angabe der Quelle kenntlich gemacht. Die Arbeit ist noch nicht veröffentlicht oder in anderer Form als Prüfungsleistung vorgelegt worden.\newline
Ich habe die Bedeutung der ehrenwörtlichen Versicherung und die prüfungsrechtlichen Folgen (§ 24 Abs. 2 Bachelor-SPO (7 Semester) der HdM) einer unrichtigen oder unvollständigen ehrenwörtlichen Versicherung zur Kenntnis genommen.


\vspace{4\baselineskip}

\begin{tabular}{lp{2em}l}
 \hspace{5cm}   && \hspace{4cm} \\
 \cline{1-1}\cline{3-3}
 Ort, Datum     && Unterschrift
\end{tabular} 

\cleardoublepage
%\pagestyle{empty}
\section*{Kurzfassung}
Netzwerkgebundene Anwendungen wie Web Server und in-memory Datenbanken bilden das Fundament für moderne stark verteilte Systeme. Die Leistung von Netzwerkkarten und der dazugehörigen Infrastruktur steigt jährlich, indessen nimmt die Leistungszunahme von \linebreak modernen Prozessoren pro Kern weiter ab. Um dieser Entwicklung entgegenzuwirken müssen diese Anwendungen horizontal und vertikal skalierbar sein. Parallele Algorithmen ermöglichen vertikale Skalierung auf modernen Mehrkern-Prozessoren. Diese Thesis betrachtet wie \linebreak netzwerkgebundene Anwendungen auf modernen Mehrkern-Prozessoren skalieren können. Die dabei behandelten Themen reichen von Hardware und Betriebssystemen bis hin zu \linebreak Algorithmen für dynamisches “Scheduling” im “User Space”.\newline
Als ein wesentlicher Teil dieser Thesis wurden Benchmark-Tests für die in-memory Datenbanken Redis (mit  “I/O Threading”), KeyDB (“Redis fork” mit einer “multithreaded Event Loop”), Mini-Redis (unvollständiger Redis Server, der von “Work-Stealing Scheduling” \linebreak Gebrauch macht) und Redis Cluster (“shared-nothing” Datenbank-Cluster) durchgeführt. Alle getesteten Datenbanken liefen auf einem (physischen) Server. Die Performance-Metriken waren der absolute Durchsatz und “Tail Latency”. In diesen Benchmarks erzielt die “shared-nothing” Datenbank Redis Cluster den höchsten Durchsatz aller getesteten Anwendungen. Im Allgemeinen weisen die Tests darauf hin, dass je weniger Daten unter Prozessen geteilt werden und je weniger Mehraufwand die Parallelverarbeitung verursacht, desto höher ist der Durchsatz den diese Anwendungen bei homogener Last maximal erzielen können. Auf der anderen Seite zeigt diese Arbeit aber auch auf, dass das “shared-nothing” Paradigma, abgesehen von anderer Nachteilen, womöglich nicht die optimalste Strategie für die Senkung von “Tail Latencies” ist. 
Das Teilen von Daten zwischen Prozessen, zum Beispiel durch die Verwendung von “Single-Queue, Multi-Server” Queueing-Modellen und dynamischen “Scheduling” Strategien, kann bezüglich der “Tail Latencies” trotz Mehraufwand bessere Resultate erzielen, vor allem bei “skewed Workloads”.

\setcounter{savepage}{\thepage}
\begin{abstractpage}
% $Log: abstract.tex,v $
% Revision 1.1  93/05/14  14:56:25  starflt
% Initial revision
% 
% Revision 1.1  90/05/04  10:41:01  lwvanels
% Initial revision
% 
%
%% The text of your abstract and nothing else (other than comments) goes here.
%% It will be single-spaced and the rest of the text that is supposed to go on
%% the abstract page will be generated by the abstractpage environment.  This
%% file should be \input (not \include 'd) from cover.tex.
Highly concurrent network-bound applications like web servers and in-memory databases are the foundation of modern large scale distributed systems. While the performance of NICs and network infrastructure increases every year, improvements in the single-core performance of modern server CPUs seem to have plummeted in the past years. 
Highly concurrent network-bound applications need to counteract by scaling horizontally \textbf{and} vertically. One way to scale vertically in these applications is to implement parallel processing for multi-core scaling on modern CPUs. This thesis takes a look at how this can be achieved and the topics discussed range from hardware and operating system related issues to algorithms for dynamic scheduling in user space. \newline
As part of this study benchmark tests were performed with the in-memory databases Redis (with I/O threading), KeyDB (Redis fork with a multithreaded event loop), Mini-Redis (incomplete Redis server that leverages work-stealing scheduling) and Redis Cluster (shared-nothing database cluster), which all ran on a single (physical) node. In the benchmarks the performance metrics tested were throughput and tail latency. The evaluated results demonstrate that the shared-nothing database Redis Cluster delivers the best results for throughput of all the tested applications. Generally, the tests indicate that the less data is shared in-between processes and the less overhead the algorithms for parallel processing introduce, the better the performance in regards to throughput in uniform load scenarios. However, this study also demonstrates that the shared-nothing approach, besides having other disadvantages, might not be the optimal strategy for lowering tail latencies. Sharing data in-between processes, e.g. by leveraging single-queue, multi-server models and dynamic scheduling strategies, which typically add overhead, can deliver better results in regards to tail latencies, especially when workloads are skewed.


\end{abstractpage}

% Additional copy: start a new page, and reset the page number.  This way,
% the second copy of the abstract is not counted as separate pages.
% Uncomment the next 6 lines if you need two copies of the abstract
% page.
% \setcounter{page}{\thesavepage}
% \begin{abstractpage}
% % $Log: abstract.tex,v $
% Revision 1.1  93/05/14  14:56:25  starflt
% Initial revision
% 
% Revision 1.1  90/05/04  10:41:01  lwvanels
% Initial revision
% 
%
%% The text of your abstract and nothing else (other than comments) goes here.
%% It will be single-spaced and the rest of the text that is supposed to go on
%% the abstract page will be generated by the abstractpage environment.  This
%% file should be \input (not \include 'd) from cover.tex.
Highly concurrent network-bound applications like web servers and in-memory databases are the foundation of modern large scale distributed systems. While the performance of NICs and network infrastructure increases every year, improvements in the single-core performance of modern server CPUs seem to have plummeted in the past years. 
Highly concurrent network-bound applications need to counteract by scaling horizontally \textbf{and} vertically. One way to scale vertically in these applications is to implement parallel processing for multi-core scaling on modern CPUs. This thesis takes a look at how this can be achieved and the topics discussed range from hardware and operating system related issues to algorithms for dynamic scheduling in user space. \newline
As part of this study benchmark tests were performed with the in-memory databases Redis (with I/O threading), KeyDB (Redis fork with a multithreaded event loop), Mini-Redis (incomplete Redis server that leverages work-stealing scheduling) and Redis Cluster (shared-nothing database cluster), which all ran on a single (physical) node. In the benchmarks the performance metrics tested were throughput and tail latency. The evaluated results demonstrate that the shared-nothing database Redis Cluster delivers the best results for throughput of all the tested applications. Generally, the tests indicate that the less data is shared in-between processes and the less overhead the algorithms for parallel processing introduce, the better the performance in regards to throughput in uniform load scenarios. However, this study also demonstrates that the shared-nothing approach, besides having other disadvantages, might not be the optimal strategy for lowering tail latencies. Sharing data in-between processes, e.g. by leveraging single-queue, multi-server models and dynamic scheduling strategies, which typically add overhead, can deliver better results in regards to tail latencies, especially when workloads are skewed.


% \end{abstractpage}

%\cleardoublepage

%\section*{Acknowledgments}

%This is the acknowledgements section.  You should replace this with your own acknowledgements.

%%%%%%%%%%%%%%%%%%%%%%%%%%%%%%%%%%%%%%%%%%%%%%%%%%%%%%%%%%%%%%%%%%%%%%
% -*-latex-*-
