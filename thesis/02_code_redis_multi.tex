\begin{lstlisting}
while !shutdown:
    handle_pending_reads_using_threads(global_read_queue)
    handle_pending_writes_using_threads(global_write_queue)
    # using stateful event notification interface e.g. epoll
    events = stateful_poll_wait(stateful_poll_fd, timeout)
    for event in events:
        if event.fd == listener_fd:
            event.fd.accept()
        if event.mask & READABLE:
            # will postpone read if I/O threads active
            connections[event.fd].handle_read_event()
        if event.mask & WRITABLE:
            # interest is registered (rarely) if write was previously blocking
            # due to level-triggered notifications
            connections[event.fd].handle_write_event()

# producer side (main thread)
handle_pending_writes_using_threads(global_write_queue):
    i = 0
    # distribute work
    for write_task in global_write_queue:
        thread_queues[i++ % thread_count].push_back(write_task)
    # "fan-out" - give the go to the worker I/O threads
    for thread_id = 0; thread < thread_count; thread_id++:
        thread_pending[thread_id].atomic_store(thread_queues[thread_id].length)
    # meanwhile main thread (id = 0) handles its slice
    while write_task = thread_queues[0].pop():
        handle_pending_write(write_task)
        thread_pending[0]--
    # "join" - busy-waiting for the other threads to finish their work
    while sum(thread_pending) != 0
\end{lstlisting}
\pagebreak
\begin{lstlisting}
# consumer side (spinning worker threads)
thread_create(
    # note: in the real implementation this thread can be blocked
    # by a mutex that is controlled by the main thread
    while True:
        # busy-waiting for write task from the queue
        while thread_pending[thread_id].atomic_load() == 0
        # thread handles its slice
        while write_task = thread_queues[thread_id].pop():
            handle_pending_write(write_task)
            thread_pending[thread_id]--
, thread_count - 1)
\end{lstlisting}